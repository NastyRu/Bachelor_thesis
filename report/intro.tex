\Introduction

Правильная организация работы с документами в наши дни имеет большое значение, так как от эффективности реализации документооборота напрямую зависит эффективность работы любой организации. Например, в рамках кредитных конвейеров юридических лиц банки запрашивают у компаний оригиналы различных документов. Для удобства использования их нужно классифицировать на отдельные документы, в том числе одностраничные или многостраничные \cite{bank}.

Во многих интернет-сервисах, от платежных систем до схем восстановления доступа к учетным записям в социальных сетях, активно используются инструменты распознавания документов. 

Например, в таких системах, как портал <<Госуслуги>>\cite{gosuslugi} и платежная система <<Webmoney>>\cite{webmoney}, целью обработки документа является получение всей персональной информации о клиенте, например, в случае паспорта РФ это фамилия, имя, серия и номер паспорта, место и дата выдачи, код подразделения, машиночитаемая зона, и установление ее подлинности. 

В социальных сетях, таких как <<ВКонтакте>>\cite{vk}, для восстановления доступа к учетной записи пользователя, не требуется получать всю персональную информацию, указанную на странице документа, удостоверяющего личность пользователя. В большинстве случаев достаточно проверить изображение определенной части документа, содержащей некоторую окрестность фотографии пользователя. Данная окрестность должна полностью содержать фотографию пользователя, имя, фамилию. 

Тип документа определяется как текстом, так и визуальной информацией. Например, паспорт или трудовую книжку легко различить визуально без анализа текста внутри. Более того, качество распознавания текста в таких документах достаточно низкое, если используются неспециализированные решения. Поэтому визуальная составляющая несет намного больше релевантной информации для классификации. Однако различные типы виз могут быть визуально похожи, однако текстовая информация, которую они содержат, отличается. 

В результате задача классификации документов сводится к модели, которая должна объединить два источника неструктурированных данных: визуальное представление документа (устойчивые признаки для проверки изображения, удостоверяющего личность человека) и результаты распознавания текстовой информации.

Целью работы является разработка и исследование метода классификации документов по фотографии. Имеется множество документов написанных на естественном языке и множество заранее известных категорий. Требуется для каждого документа выбрать категорию, к которой он, в силу своего смыслового (семантического) содержания, относится с наибольшей долей уверенности.

Для достижения поставленной цели необходимо решить следующие задачи.
\begin{itemize}
\item Анализ существующих решений.
\item Разработка метода классификации.
\item Спроектировать и реализовать ПО, демонстрирующее работу метода.
\item Провести исследование на применимость данного метода, его соответствие цели работы.
\end{itemize}
